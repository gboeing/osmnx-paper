\RequirePackage[l2tabu,orthodox]{nag}     % warn if using any obsolete or outdated commands
\documentclass[12pt,letterpaper]{article} % document style

% import encoding and font packages for pdflatex, in order
\usepackage[T1]{fontenc}    % output T1 font encoding (8-bit) so accented characters are a single glyph
\usepackage[utf8]{inputenc} % allow input of utf-8 encoded characters
\usepackage{ebgaramond}     % document's serif font
\usepackage{tgheros}        % document's sans serif font

% import language and regionalization packages, in order
\usepackage[strict,autostyle]{csquotes} % smart and nestable quote marks
\usepackage[USenglish]{babel}           % automatically regionalize hyphens, quote marks, etc
\usepackage{microtype}                  % improves text appearance with kerning, etc

% import everything else
\usepackage{abstract}                   % allow full-page title/abstract in twocolumn mode
\usepackage{authblk}                    % footnote-style author/affiliation info
\usepackage{booktabs}                   % better looking tables
\usepackage{caption}                    % custom figure/table caption styles
\usepackage{datetime} % enable formatting of date output
\usepackage[final]{draftwatermark}      % watermark paper as a draft
\usepackage{endnotes}                   % enable endnotes
\usepackage{geometry}                   % configure page dimensions and margins
\usepackage{graphicx}                   % better inclusion of graphics
\usepackage{hyperref}                   % hypertext in document
\usepackage{natbib}                     % author-year citations w/ bibtex, including textual and parenthetical
\usepackage{rotating}                   % rotate wide tables or figures on a page to make them landscape
\usepackage{setspace}                   % configure spacing between lines
\usepackage{titlesec}                   % custom section and subsection heading
\usepackage{url}                        % make nice line-breakble urls

% print only the month and year when using \today
\newdateformat{monthyeardate}{\monthname[\THEMONTH] \THEYEAR}

\newcommand{\myname}{Geoff Boeing}
\newcommand{\myemail}{boeing@usc.edu}
\newcommand{\myaffiliation}{University of Southern California}
\newcommand{\paperdate}{\monthyeardate\today}
\newcommand{\papertitle}{Modeling and Analyzing Urban Networks and Features with OSMnx}
\newcommand{\papercitation}{Boeing, G. \the\year. \papertitle.}
\newcommand{\paperkeywords}{geography, openstreetmap, network science, street networks, transportation, urban planning}


% location of figure files, via graphicx package
\graphicspath{{./figures/}}

% configure the page layout, via geometry package
\geometry{
	paper=letterpaper,         % paper size
	top=3.8cm,                   % margin sizes
	bottom=3.8cm,
	left=4cm,
	right=4cm}
\setstretch{1}              % line spacing
\clubpenalty=10000             % prevent orphans
\widowpenalty=10000            % prevent widows

% set section/subsection headings as the sans serif font, via titlesec package
\titleformat{\section}{\normalfont\sffamily\large\bfseries\color{black}}{\thesection.}{0.3em}{}
\titleformat{\subsection}{\normalfont\sffamily\small\bfseries\color{black}}{\thesubsection.}{0.3em}{}

% make figure/table captions sans-serif small font
\captionsetup{font={footnotesize,sf},labelfont=bf,labelsep=period}

% configure pdf metadata and link handling, via hyperref package
\hypersetup{
	pdfauthor={\myname},
	pdftitle={\papertitle},
	pdfsubject={\papertitle},
	pdfkeywords={\paperkeywords},
	pdffitwindow=true,         % window fit to page when opened
	breaklinks=true,           % break links that overflow horizontally
	colorlinks=false,          % remove link color
	pdfborder={0 0 0}          % remove link border
}
%\doublespacing
\begin{document}

\title{\papertitle}
\author[]{\myname}
\affil[]{\myaffiliation}
\date{\paperdate}

\maketitle

\begin{abstract}

OSMnx is a Python package for downloading, modeling, analyzing, and visualizing urban networks and any other geospatial features from OpenStreetMap data. A large and growing body of literature has used it to conduct scientific studies across the disciplines of geography, urban planning, transport engineering, computer science, and others. OSMnx has recently seen the development and implementation of many new features, modeling capabilities, and analytical methods. The package now encompasses much more functionality that was previously reported in the literature. This paper accordingly introduces the OSMnx package's new capabilities, organization, and usage---in addition to the scientific concepts and logic underlying them. Finally, it reflects on tool building in geographical information science and the implications for urban modeling and analysis in the mode of open science.\footnote{Citation info: \papercitation~Correspondence: \href{mailto:\myemail}{\myemail}}

\end{abstract}

\section{Introduction}

Modeling networked urban infrastructure---such as streets, rails, canals, etc.---and geospatial features---such as amenities, points of interest, etc.---underpins research into travel behavior, accessibility, public health, sustainability, and spatial equity. This research spans the disciplines of urban morphology \citep[e.g.,][]{gervasoni_calculating_2017,dacci_signature_2019,coutrot_cities_2020}, transportation planning \citep[e.g.,][]{merchan_quantifying_2020,liao_disparities_2020,natera_orozco_data-driven_2019}, and network science \citep[e.g.,][]{feng_spatial_2020,yin_multi-task_2020,young_automatic_2020}. However, the traditional limitations of data availability, inconsistent digitization standards, and lack of well-documented, reusable tools limited the reproducibility, generalizability, scalability, and usefulness of empirical urban network modeling \citep{liu_generalized_2021}. Urban researchers need better tools for better urban science.

This paper describes OSMnx, a Python package that allows users to easily download, model, analyze, and visualize urban networks and geospatial features from OpenStreetMap data. It enables users to download and model walking, driving, biking, or custom networks with a single line of code and then analyze and visualize them. Users can easily work with urban amenities and points of interest, building footprints, transit stops, elevation data, street orientations, speed and travel time, and routing.

The OSMnx software was first released in a beta version in 2016, and this early incarnation was documented in an accompanying article \citep{boeing_osmnx:_2017}. Over the intervening years, the package evolved substantially, expanding in functionality, improving performance, and stabilizing a user-friendly API. It has since become a widespread tool for urban modeling and analysis \citep{boeing_right_2020}. This paper documents OSMnx 2.0's new functionality, including its modular organization and uses for urban modeling and analysis.

This paper is organized as follows. The next section provides a brief introduction to street network modeling and analysis, including the current tool landscape and OSMnx's place in it. Then it presents the OSMnx package's functionality, organization, and usage. Finally it concludes with a brief discussion of implications for urban network analysis and tool building in geographical information science.

\section{Analyzing Street Networks}

Numerous sources cover spatial network theory and analysis \citep[e.g.,][]{tinkler_graph_1979,barnes_graph_1983,gastner_spatial_2006,barthelemy_spatial_2011,ducruet_spatial_2014,fischer_spatial_2014,marshall_street_2018}. This section focuses on street networks and offers a brief overview of key concepts and the current tool landscape.

\subsection{Street Network Models}

Real-world networks are commonly modeled as mathematical \textit{graphs} \citep{trudeau_introduction_1994}. In computer science, a graph, $G$, is a data structure comprising a set, $N$, of \textit{nodes} linked to each other by a set, $E$, of node pairs called \textit{edges}. Given $G = (N, E)$ and $\{u, v\} \subseteq N$ and $\{u, v\} \in E$, then we can say for graph $G$ that: 1) edge $\{u, v\}$ links nodes $u$ and $v$, 2) $\{u, v\}$ is \textit{incident} on $u$ and on $v$, 3) $u$ is \textit{adjacent} to $v$ and vice versa, and 4) $u$ and $v$ are \textit{neighbors} \citep{newman_networks:_2010}. An adjacency matrix can fully represent a graph by defining adjacent node pairs.

A node's \textit{degree} is how many edges are incident on that node. For example, a node with degree 2 has two incident edges which are consequently adjacent to each other. An edge can be \textit{directed} (linking one node to another node one-way), \textit{undirected} (linking two nodes bidirectionally), or a \textit{self-loop} (linking one node to itself). A graph with directed edges is a directed graph, or \textit{digraph}. A graph that allows multiple edges to link a single pair of nodes is a \textit{multigraph} and those multiple edges are called \textit{parallel} edges.

A \textit{path} is a sequence of edges linking a sequence of nodes. The graph \textit{distance} between two nodes is the count of edges in the shortest such path between them. A \textit{weighted} graph's distance is the sum of some edge impedance attribute (e.g., length or time) along the shortest path minimizing that sum. A directed graph is strongly \textit{connected} if a path exists between each ordered pair of nodes and it is weakly connected if such a path exists only if its edges are undirected. A disconnected graph contains multiple connected \textit{components}---each a disjoint set of nodes forming its own connected subgraph.

Spatial networks' nodes and/or edges are embedded in space. Spatial graphs thus model both \textit{topology} and \textit{geometry} \citep{fischer_spatial_2014}. Topology refers to the structure and configuration of the nodes and edges, whereas geometry encompasses positions, lengths, angles, etc. A \textit{planar} graph can be represented in a two-dimensional plane such that its edges intersect only at nodes \citep{barthelemy_modeling_2008,barthelemy_spatial_2011}. Most street networks are nonplanar due to the presence of bridges, tunnels, overpasses, and underpasses, but their spatial embedding and thus approximate planarity constrain their topological characteristics relative to other kinds of complex networks \citep{boeing_planarity_2020}. A \textit{primal} graph of a street network models its intersections and dead-ends as nodes and its street segments as edges \citep{porta_network_2006-1}. A dual or line graph inverts this topology, modeling streets as nodes and their intersections as edges \citep{porta_network_2006}, though this discards most of the network's geographical characteristics \citep{ratti_space_2004}.

\subsection{Street Network Analysis}

Many common geometric measures of spatial networks are often applied to street networks. These usually use undirected representations of the graph to avoid double-counting bidirectional streets relative to one-ways. Intersection density (i.e., the count of nodes with degree >1, normalized by network area) is perhaps the most common such measure of network \enquote{grain} in transport planning and urban design \citep[e.g.,][]{ewing_travel_2010}. The average street segment length (i.e., mean edge length) offers a linear proxy of block size. Street density is the sum of edge lengths normalized by network area. Average circuity is the sum of edge lengths divided by the sum of great-circle distances between adjacent node pairs, indicating the spatial inefficiency of the network as the inverse of straightness \citep{boeing_urban_2019}.

Additionally, many topological measures in network science are often applied to street networks. The \textit{average node degree} (i.e., mean number of edges incident on the nodes) indicates graph \textit{connectedness}\endnote{The separate term \textit{connectivity} has a distinct definition in graph theory, but is less useful for spatial network analysis because approximate planarity sharply constrains it: almost all street networks have connectivity equal to 1.} and is perhaps the most common topological measure in transport planning and urban design \citep{barrington-leigh_century_2015,barrington-leigh_more_2017,barrington-leigh_global_2020}. Networks with high connectedness can be more robust against perturbation as they offer alternate routing options if parts of the network fail. Networks with low connectedness or chokepoints are more brittle and vulnerable to perturbation as they force circulation through single points of failure. Various measures of \textit{centrality} are also common \citep{crucitti_centrality_2006}. For example, a node's \textit{betweenness centrality} measures the share of all graph shortest paths that pass through the node \citep{barthelemy_betweenness_2004,barthelemy_self-organization_2013}. The graph's maximum betweenness centrality indicates the share of shortest paths that rely on its most important node: high values suggest possible chokepoints that represent single points of failure, such as a bridge connecting a city's halves across a river \citep{boeing_multi-scale_2020}.

In street network analysis, researchers use these geometric and topological measures to characterize a network's form. Such analyses often also employ path solving \citep{miller_measuring_1999,wang_road_2020,liu_generalized_2021}. For example, accessibility analyses solve shortest paths (e.g., by length or travel time) from origin nodes (e.g., homes) to destination nodes representing the locations of amenities (e.g., workplaces, schools, transit stops, parks, food markets, health care, etc.) to measure access. Disaster analyses often simulate evacuations along the network to understand the flows leaving a city following a disaster. These kinds of analyses can reveal differential outcomes for different communities or locations with the city, offering guidance for practitioners' interventions.

\subsection{Street Network Data and Tools}

Several tools exist to model and analyze spatial networks like street networks. For example, ESRI's ArcGIS software includes a Network Analyst extension and QGIS offers plug-ins for network analysis. However, such GIS tools' network capabilities are usually fairly limited. Conversely, dedicated network analysis tools such as Gephi, igraph, graph-tool, and NetworkX offer robust network analysis functionality but limited geospatial capabilities. Other dedicated spatial network tools exist, often for specific analytical purposes, including PySAL spaghetti for network inference \citep{gaboardi_spaghetti_2021,rey_pysal_2022}, Pandana for accessibility analysis \citep{foti_behavioral_2014}, momepy for urban morphology \citep{fleischmann_momepy_2019}, and stplanr \citep{lovelace_stplanr_2019}.

Street network data come from various sources, such as the US Census Bureau's TIGER/Line shapefiles which represent network geometry but lack thorough topological representations. OpenStreetMap offers a worldwide, public web mapping platform and geospatial database that anyone contribute to, with some editorial oversight \citep{jokar_arsanjani_openstreetmap_2015}. Although its coverage varies, it offers street network geometry and topology and the data quality is generally high---particularly so in urban areas, with notable exceptions in China and sub-Saharan Africa \citep{barron_comprehensive_2014,barrington-leigh_worlds_2017}. OpenStreetMap's data model comprises three \textit{element} types: \textit{nodes} (points), \textit{ways} (either open ways representing lines or closed ways representing polygons), and \textit{relations} (i.e., between nodes and/or ways). These elements can possess one or more \textit{tags}: key-value pairs containing attribute data. OpenStreetMap data are available to download from third-party services, such as GeoFabrik, and web APIs including Overpass and Nominatim.

OSMnx fits into this landscape by allowing users to automatically build a street (or other) network model anywhere in the world by downloading raw data from OpenStreetMap to build a NetworkX model with spatial information. It fills several needs, as discussed in the following section. Accordingly, it has become a standard tool for both retrieving OpenStreetMap data and for modeling street networks. This following section provides a current introduction to the package's organization, capabilities, and usage.

\begin{table*}[tbp]
	\centering
	\caption{The OSMnx public API's modules and their functionality.}
	\label{tab:my_table}
	\begin{tabular}{lp{10cm}}
		\toprule
		Module &  Functionality \\
		\midrule
		\texttt{bearing}  & Calculate graph edge compass bearings and orientation entropy. \\
		\texttt{convert}  & Convert graph to/from different data types. \\
		\texttt{distance}  & Calculate spatial distances and find nearest graph node/edge(s) to point(s). \\
		\texttt{elevation}  &  Attach node elevations from raster files or a Google Maps compatible elevation API, and calculate edge grades. \\
		\texttt{features}  &  Download OSM geospatial features' geometries and attributes, such as points of interest, building footprints, transit stops, etc. \\
		\texttt{geocoder}  & Geocode place names or addresses or retrieve OSM elements by place name or ID, via the Nominatim API. \\
		\texttt{graph}  & Download and create graphs from OSM data, using filters to query the Overpass API for built-in network types or a custom filter. \\
		\texttt{io}  & Save/load graphs to/from GraphML, GeoPackage, or OSM XML files. \\
		\texttt{plot} & Visualize street networks, routes, orientations, and geospatial features.  \\
		\texttt{projection}  &  Project spatial graph to a different coordinate reference system. \\
		\texttt{routing}  &  Calculate graph edge speeds, travel times, and weighted shortest paths between nodes. \\
		\texttt{settings}  & Configure global package settings. \\
		\texttt{simplification}  & Simplify and consolidate spatial graph nodes and edges. \\
		\texttt{stats}  &  Calculate geometric and topological network measures. \\
		\texttt{truncate}  & Truncate spatial graph by distance, bounding box, or polygon. \\
		\texttt{utils}  & General utility functions. \\
		\texttt{utils\_geo}  & Miscellaneous geospatial utility functions. \\
		\bottomrule
	\end{tabular}
\end{table*}

\section{The OSMnx Package}

OSMnx (pronounced as the initialism: \enquote{oh-ess-em-en-ex}) is a fully type-hinted Python package. It is built on top of and uses the data structures of NetworkX (a network analysis Python package) and GeoPandas (a Python package for working with spatial dataframes). It interacts with three public web APIs to collect data: the OpenStreetMap Nominatim API, the Overpass API, and the Google Maps Elevation API (or equivalent API with the same interface). This section summarizes the package's key functionality and organization.

\subsection{Geocoding and Querying}

OSMnx geocodes place names and addresses with the Nominatim API. Users can use the \texttt{geocoder} module to geocode place names or addresses to latitude-longitude point coordinates. Or, they can retrieve place boundaries or any other OpenStreetMap elements by name or ID. Using the \texttt{features} and \texttt{graph} modules, as described below, users can download data from the Overpass API by latitude-longitude point coordinates, address, bounding box, bounding polygon, or place name (e.g., neighborhood, city, county, etc.).

\subsection{Urban Amenities}

Using OSMnx's \texttt{features} module, users can search for and download any geospatial features (e.g., building footprints, grocery stores, schools, public parks, transit stops, etc.) from the OpenStreetMap Overpass API as a GeoPandas \texttt{GeoDataFrame} object. This uses OpenStreetMap tags to search for matching elements.

\subsection{Modeling Networks}

OSMnx models spatial networks as primal, nonplanar, weighted, directed multigraphs with possible self-loops---specifically, these are NetworkX \texttt{MultiDiGraph} data structures \citep{hagberg_exploring_2008}. Using OSMnx's \texttt{graph} module, users can download any spatial network data (such as streets, paths, rail, canals, power lines, etc.) from the Overpass API and model them as a \texttt{MultiDiGraph}.

OSMnx models a one-way street as a single directed edge from node $u$ to node $v$, but a bidirectional street is modeled with two reciprocal directed edges (with identical geometries)---one from $u$ to $v$ and another from $v$ to $u$---to represent both possible directions of flow. Because these graphs are nonplanar, they correctly model the topology of interchanges, overpasses, and underpasses. That is, edge crossings in a two-dimensional plane are not intersections in an OSMnx model unless they represent true junctions in the three-dimensional real world.

The \texttt{graph} module uses filters to query the Overpass API: users can either specify a built-in network type or provide their own custom filter written in OverpassQL. Under the hood, OSMnx does several things to generate the best possible model. It initially buffers the query area by 500 meters to create the initial graph before truncating it to the user's desired query area. This ensures accurate streets-per-node counts by attenuating graph perimeter effects. Anything else? OSMnx also automatically simplifies the graph topology as discussed below.

\subsection{Topology Correction and Simplification}

The \texttt{simplification} module automatically processes the network's topology from the original raw OpenStreetMap data to ensure that individual nodes represent individual intersections or dead-ends and edges represent the street segments that link them. This simplification is of two primary types: graph simplification and intersection consolidation.

\textit{Graph simplification}, put simply, merges adjacent edges for a better model. It cleans up the graph's topology so that nodes represent intersections or dead-ends and edges represent street segments. This is important because in OpenStreetMap's raw data, ways comprise sets of straight-line segments between nodes: that is, nodes are vertices for streets' curving line geometries, not just intersections and dead-ends. By default, OSMnx simplifies this topology by deleting non-intersection/dead-end nodes, merging the edges between them into a new \enquote{simplified} edge, and retaining the complete true edge geometry as an edge attribute. When multiple OpenStreetMap ways are thus merged into a new graph edge, the ways' attribute values can be aggregated into a single value.

\textit{Intersection consolidation}, put simply, merges nearby nodes for a better model. This is important because many real-world street networks feature complex intersections and traffic circles, resulting in a cluster of graph nodes where there is really just one true intersection as we would consider it in transport planning or urban design. Similarly, divided roads are often represented by separate centerline edges. The intersection of 2 divided roads thus creates 4 nodes where each edge intersects a perpendicular edge---but these 4 nodes represent a single intersection in the real world. OSMnx can consolidate such complex intersections into a single node and optionally rebuild the graph's edge topology accordingly. When multiple OpenStreetMap nodes are thus merged into a new graph node, the nodes' attribute values can be aggregated into a single value.

Graph simplification and intersection consolidation offer several benefits. They produce a more accurate model that better represents the real world. This in turn generates more accurate network stats, for example by not overcounting complex intersections when calculating intersection density or by not underrepresenting street segment lengths. Finally, many graph algorithms' time complexity scales with node or edge count. By generating a graph with (often drastically) fewer nodes and edges---yet no loss of accuracy---many algorithms will complete much faster. This matters most when analyzing large urban networks, where runtime becomes an issue.

\subsection{Converting, Projecting, and Saving}

OSMnx's \texttt{convert} module can convert a NetworkX \texttt{MultiDiGraph} model to a NetworkX \texttt{MultiGraph} if the user prefers an undirected representation of the network for specific analytical purposes, as discussed in the background section. It can also convert to a NetworkX \texttt{DiGraph} if the user prefers a directed graph without any parallel edges. OSMnx can also convert a \texttt{MultiDiGraph} to and from node and edge GeoPandas \texttt{GeoDataFrame} objects. The resulting nodes \texttt{GeoDataFrame} is indexed by OpenStreetMap node ID, and the resulting edges \texttt{GeoDataFrame} is multi-indexed by endpoint node $u$, endpoint node $v$, and \textit{key} (to differentiate parallel edges), just as a \texttt{MultiDiGraph} edge is identified by a $u$, $v$, \textit{key} ordered triplet. This also allows users to load arbitrary node and edge ShapeFiles or GeoPackage layers as \texttt{GeoDataFrame}s then convert them to a \texttt{MultiDiGraph} for network analysis.

As these models are all spatial graphs, they have coordinate reference system (CRS) metadata. OSMnx's default CRS is EPSG:4326, but users can project a graph to any other CRS using the \texttt{projection} module. If a user is unsure which CRS to project to, OSMnx can automatically determine an appropriate Universal Transverse Mercator CRS for the operation, based on the graph nodes' centroid.

Finally, using the \texttt{io} module, users can save a graph to disk as a GraphML file (to load into other network analysis software), a GeoPackage (to load into other GIS software), or an OSM XML file (the standard OpenStreetMap data interchange format).

\subsection{Elevation}

Topography is essential to understanding street network form, but street network analyses too often ignore elevation as it can be difficult to acquire and attach to a graph model \citep{boeing_street_2021}. OSMnx's \texttt{elevation} module lets users automatically add elevation attributes to a spatial graph's nodes from either a local raster file or the Google Maps Elevation API (or an alternative API with the same interface). Once all nodes have elevation data, users can calculate edge grades (i.e., rise-over-run inclines), analyze the steepness of certain streets, or use elevation change in an impedance function for routing, as discussed below.

\subsection{Map Matching and Routing}

OSMnx offers basic map matching and routing functionality. The \texttt{distance} module can match the nearest node or edge to each of a set of coordinates using a fast spatial index and vectorized operation. For example, this can be useful for converting an origin-destination matrix of coordinates (such as geocoded addresses) into corresponding nearest nodes to solve routes between them.

The \texttt{routing} module can solve shortest paths for network routing---parallelized with multiprocessing---using different weights (e.g., distance, travel time, elevation change, etc.). OpenStreetMap has a \enquote{maxspeed} tag representing streets' maximum speed limits, but it tends to be sparse for most cities. To address this problem, the \texttt{routing} module can impute missing edge maximum speeds based on observed values across other edges of the same type in the graph. Such imputation can be imprecise, but the user can override it by passing per-type local speed limits. Once all edges have maximum speed attributes, the module can also automatically calculate free-flow traversal times for each edge.

\subsection{Network Measures}

Users can use the \texttt{stats} module to calculate a variety of geometric and topological measures of the network \citep{boeing_street_2021}. These measures define streets as the edges in an undirected representation of the graph to prevent double-counting the bidirectional edges of a two-way street. Users can automatically calculate common measures from transport planning, urban design, and network science, including intersection density, circuity, average node degree, betweenness centrality, and many others. Users can also use NetworkX directly to calculate additional topological network measures. OSMnx's \texttt{bearing} module can calculate the streets' compass bearings and its orientation entropy.

\subsection{Visualization}

Users can plot graphs, routes, network figure-ground diagrams, building footprints, and street network orientation polar histograms \citep{boeing_urban_2019,boeing_spatial_2021} using OSMnx's \texttt{plot} module. Users can also easily explore street networks, routes, or urban amenities as interactive web maps.

\subsection{Installation and Configuration}

Users can install the OSMnx package from the PyPI package repository using pip or from the Anaconda package repository using conda, as detailed in the package documentation\endnote{The OSMnx documentation is available at \href{https://osmnx.readthedocs.io/en/latest/}{https://osmnx.readthedocs.io/en/latest/}} and its installation instructions. Once installed and imported, users can configure OSMnx using its \texttt{settings} module. Here they can adjust logging behavior, server response caching, server endpoints (including pointing to locally hosted instances), and much more. Users can also configure OSMnx to retrieve historical snapshots of OpenStreetMap data as of a certain date.

%\section{Discussion}


%\section*{Acknowledgments}

% print the footnotes as endnotes, if any exist
\IfFileExists{\jobname.ent}{\theendnotes}{}

% print the bibliography
\setlength{\bibsep}{0.00cm plus 0.05cm} % no space between items
\bibliographystyle{apalike}
\bibliography{references}

\end{document}
